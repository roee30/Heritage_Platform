The function {\sl react1} is a recognizer for the rational language which is the 
image by the {\sl transducer} morphism of the regular expression over phases. 
It stops at the first solution - when the input string is a word in this language 
- otherwise it raises the exception {\sl Finished}. However, note that the 
general mechanism for managing non-determinism through coroutine resumptions 
allows restarting the computation to find other solutions. This mechanism will 
be specially important later when our engine
is used for transductions, where we may be interested in the various solutions.

We give above an example of using {\sl continue} as a coroutine by computing 
the {\sl multiplicity} function, which counts the number of ways in which 
the input string may be solution to the regular expression. We remark that 
standard formal language theory deals with languages as {\sl sets} of words,
whereas here we formalize the finer notion of a {\sl stream} (i.e. a 
potentially infinite list) of words recursively enumerating a {\sl multiset}
of words. 

% amiable together etc -> 4 TODO

% completeness, finiteness condition : morphism such that epsilon not in languages 

\section{Modular aum transducers}

So far our automata were mere deterministic recognizers for finite sets of words
(although a dose of non-determinism arises from the choice, at any accepting node,
between external transition to the next phase and continuing the local search, since
the local language may contain a word and one of its proper prefixes). 
We now consider a more general framework where we handle loops in the transition 
relation, non-deterministic transitions, and output.

