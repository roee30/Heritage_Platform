%Part II
\part{Reactive Transducers}

\section{Introduction}

This second part gives additional tools for manipulating variants of finite-state machines.
They are a natural extension of the unglueing process presented at the end of Part I.

The general idea is to represent applicatively the state graph of finite-state machines
as a decorated dictionary. The dictionary, used as spanning tree of the state transition graph,
is a deterministic subset of this graph. The rest of the structure of the finite-state machine,
permitting the representation of non-determinism, of loops, and of transducer operations,
is encoded as attributes decorating the dictionary nodes. This general framework
of {\sl mixed automata} or {\sl aums}, is described in reference \cite{2003-Huet-3}. 
Its application to the problem of segmentation and tagging of Sanskrit is described
in \cite{2004-Huet-1}. 

We provide here various specific examples of this general methodology, and a mechanism
for composing such finite-state descriptions in a modular fashion \cite{2006-Huet-Razet}.

This methodology has been lifted more recently
to a very general paradigm of relational 
programming within the framework of Eilenberg machines by Beno{\^\i}t Razet
\cite{2008-Huet-Razet, Razet08a, Razet08b, Razet09}.

\section{A simplistic modular Automaton recognizer}

The simplest aum structure is the one reduced to deterministic acyclic finite-state automata,
where the aum structure is reduced to the underlying dictionary (Trie). Provided all
states are accessible from the initial one, the reduced structure obtained by applying the
Sharing functor yields the minimal deterministic automaton. This framework applies to the
simple but important subcase of finite languages.

We assume known the modules of the first part of the toolkit documentation. 

\subsection{Simplistic aums}

