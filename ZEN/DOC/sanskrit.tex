A list of 138257 French words, represented as an 8-bit ASCII file of 1.52
Mbytes, is represented as a trie of 2.4 Mbytes, which shrinks to 450
Kbytes by sharing. 

\subsection{Lexicon repositories using tries and decos}

In a typical computational linguistics application, grammatical information
(part of speech role, gender/number for substantives, valency and other
subcategorization information for verbs, etc) may be stored as 
decoration of the lexicon of roots/stems. From such a decorated trie
a morphological processor may compute the lexmap of all inflected forms,
decorated with their derivation information encoded as an inverse map.
This structure may itself be used by a tagging processor 
to construct the linear representation of a 
sentence decorated by feature structures. Such a representation
will support further processing, such as computing
syntactic and functional structures, typically as solutions of
constraint satisfaction problems. 

Let us for example give some information on the indexing structures 
\verb|trie|, \verb|deco| and \verb|lexmap| used
in our computational linguistics tools for Sanskrit.

The main component in our tools is a structured lexical database,
described in \cite{dico-report,wcre}. From this database,
various documents may be produced mechanically, such as a
printable dictionary through a {\TeX}/Pdf 
compiling chain,
and a Web site (\url{http://pauillac.inria.fr/~huet/SKT}) 
with indexing tools. The index CGI engine searches the words
by navigating in a persistent trie index of stem entries. In the current 
version, the database comprises 12000 items.
The corresponding trie (shared as a dag) has a size of 103KB. 

When computing this index, another persistent structure is created. It
records in a deco all the genders associated with a noun entry
(nouns comprise substantives and adjectives, a blurred distinction in
Sanskrit). 
At present, this deco records genders for 5700 nouns, and it has
a size of 268KB. 

A separate process may then iterate on this genders structure a 
grammatical engine, which for each stem and associated gender generates
all the corresponding declined forms. Sanskrit has a specially 
prolific morphology, with 3 genders, 3 numbers and 7 cases. The grammar
rules are encoded into 84 declension tables, and for each declension 
suffix an internal sandhi computation is effected to compute the final 
inflected form. All such words are recorded in a inflected forms lexmap,
which stores for every word the list of
pairs (stem,declension) which may produce it. This lexmap records
about 120000 such inflected forms with associated grammatical
information, and it has a size of 341KB (after minimization by sharing, 
which contracts approximately by a factor of 10). 
A companion trie, without the information, keeps the index of inflected words 
as a minimized structure of 140KB. 

A similar process produces the conjugated forms of root verbs. 

