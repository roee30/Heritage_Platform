\section{Trie Structures for Lexicon Indexing}

Tries are tree structures that store finite sets of strings sharing
initial prefixes. 

\subsection{Tries as Lexical Trees}

Tries (also called {\sl lexical trees})
may be implemented in various ways. A node in a trie represents a
string, which may or may not belong to the set of strings encoded in the
trie, together
with the set of tries of all suffixes of strings in the set having this 
string as a prefix. The forest of sibling tries at a given level 
may be stored as an array, or as a list if we assume a sparse representation. 
It could also use any of the more efficient representations of finite sets,
such as search trees \cite{bentley}.
Here we shall assume the simple sparse representation with 
lists (which is actually the original presentation of tries
by Ren\'e de la Briantais (1959)), 
yielding the following inductive type structure.

